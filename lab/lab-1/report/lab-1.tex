\documentclass[11pt,a4paper]{article}

\usepackage{graphicx}
\usepackage{amsfonts} 
\usepackage{amssymb}
\usepackage{amsmath}
\usepackage[a4paper,margin=3cm]{geometry}
\usepackage{lastpage}
\usepackage{wrapfig}
\usepackage{matlab-prettifier}
\usepackage[colorlinks = true,
            linkcolor = blue,
            urlcolor  = blue]{hyperref}

\newcommand{\changeurlcolor}[1]{\hypersetup{urlcolor=#1}}   

\usepackage{awesomebox}
\usepackage{fancyhdr}

\usepackage[australian]{babel}
\usepackage{datetime}

\newcommand{\titlestr}{Laboratory 1: Input Interfacing with the MSP432 Using Interrupts}
\newcommand{\authorstr}{Jack Lukomski \& Zack Peters}
\pagestyle{fancy}
\fancyhf{}
%\rhead{\authorstr}
\lhead{\titlestr}
\rfoot{\begin{center}
     \thepage
\end{center}}

\definecolor{mGreen}{rgb}{0,0.6,0}
\definecolor{mGray}{rgb}{0.5,0.5,0.5}
\definecolor{mPurple}{rgb}{0.58,0,0.82}
\definecolor{backgroundColour}{rgb}{0.95,0.95,0.92}

\lstdefinestyle{CStyle}{
    backgroundcolor=\color{backgroundColour},   
    commentstyle=\color{mGreen},
    keywordstyle=\color{magenta},
    numberstyle=\tiny\color{mGray},
    stringstyle=\color{mPurple},
    basicstyle=\footnotesize,
    breakatwhitespace=false,         
    breaklines=true,                 
    captionpos=b,                    
    keepspaces=true,                 
    numbers=left,                    
    numbersep=5pt,                  
    showspaces=false,                
    showstringspaces=false,
    showtabs=false,                  
    tabsize=2,
    language=C
}

\begin{document}
\begin{titlepage}
  \centering
  \includegraphics[width=0.35\textwidth]{images/GVSU.jpg}

  \vspace{1cm}
  {\LARGE \bf{\titlestr} \par}
  
  \vspace{.5cm}
  {\LARGE {EGR-326} \par}

  \vspace{1cm}
  {\Large \authorstr \par}

  \vspace{1cm}
  \today    

  \vfill
\end{titlepage}

\newpage

\tableofcontents
\newpage

\section{Objective}
The purpose of this laboratory is to develop a program for the MSP432 microcontroller that 
interfaces with pushbutton switches to control a red-green-blue (RGB) light-emitting diode (LED),
utlize a debouncing method to have percice button inputs, and to use multiple pushbuttons to carry 
out functions.

\section{Materials}

\subsection{Apparatus}

\begin{enumerate}
  \item Digital Multi-Meter
  \item DC Power Supply
\end{enumerate}

\subsection{Components}

\begin{enumerate}
  \item \(2x\) Pushbuttons
  \item \(100\Omega\) Resistor
  \item Common Cathode RGB LED
  \item MSP432R Microcontroller
  \item Breadboard
  \item Assorted Wires
\end{enumerate}

\subsection{Software}

\begin{enumerate}
  \item Code-Composer Studio (CCS)
\end{enumerate}

\section{Libraries}
Before we started creating the functional part of this laboratory, MSP432 libaries were created to make 
interfacing with the MSP432 much easier in the future.
\subsection{pinsInit Library}
\subsubsection{Purpose}
The purpose of this library is to create functions to make initlizing pins as GPIOs with pullup or pulldown
resistors much easier by calling one simple function.

\subsubsection{Header File}
\begin{lstlisting}[style=CStyle]
  /*
  * pinsInint.h
  *
  *  Created on: Sep 2, 2022
  *      Author: jtluk
  */
 
 #ifndef PINSININT_H_
 #define PINSININT_H_
 #include <stdint.h>
 #include "msp.h"
 
 typedef enum port2Pins_t
 {
     pin0,
     pin1,
     pin2,
     pin3,
     pin4,
     pin5,
     pin6,
     pin7,
 }port2Pins_t;
 
 typedef enum port2IO_t
 {
     input,
     output,
 }port2IO_t;
 
 typedef enum port2GPIOConfig_t
 {
    pullup,
    pulldown,
 }port2GPIOConfig_t;
 
 typedef struct port2GPIO_t
 {
     port2Pins_t e_IOpinNumber;
     port2GPIOConfig_t e_GPIOType;
     port2GPIOConfig_t e_IO;
 }port2GPIO_t;
 
 void vpinsInit_GPIO(port2GPIO_t * s_userGPIO_ptr, port2Pins_t e_userPin, port2IO_t e_userPortIO, port2GPIOConfig_t e_userGPIO);
 static void vPrv_pinsInit_InitIO(port2Pins_t e_userPin, port2IO_t e_userPortIO);
 static void vPrv_pinsInit_InitConfig(port2Pins_t e_userPin, port2GPIOConfig_t e_userGPIO);
 
 #endif /* PINSININT_H_ */
 
\end{lstlisting}

\section{Procedure}
\subsection{Part I – Sequencing colors of a RGB LED using a pushbutton switch}
\subsection{Description}
Part one of this laboratory consisted of creating a program to control a RGB LED using the 
MSP432R microcontroller and a pushbutton for user input. On reset, all LEDs should be off. 
On the first button press, the \textcolor{red}{red} LED should be turned on, on the second press,
the \textcolor{green}{green} LED should turn on and the \textcolor{red}{red} LED should turn off.
on the third button press, the \textcolor{blue}{blue} LED should turn on and the \textcolor{green}{green}
LED should turn off, after the forth button press, the cycle continues starting back at the \textcolor{red}{red}
LED.
\subsubsection{}

\subsection{Part II – Sequencing colors of a RGB LED using two switches with reverse direction}

\section{Code}

\section{Conclusion}

\end{document}
